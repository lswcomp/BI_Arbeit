\documentclass[%
  pdftex,%            Weiterreichen des gewählten Backend-Treibers für pdfTeX %
  a4paper,
  11pt,%                                          Größe der Grundschrift 11pt %
  oneside,%                                                zweiseitiger Druck %
  BCOR = 1cm,%                                                 Bindekorrektur %
  cleardoublepage=plain,%               Kapitelanfang immer auf rechter Seite %
  fleqn,%                                       Gleichungen links, eingerückt %
  numbers=noenddot,%                         Kapitel-Nummerierung ohne Punkte %
  headings=normal,%                                 normalgroße Überschriften %
  bibliography=totocnumbered,%                  Literaturverszeichnis ins TOC %
  listof=numbered,%               Abbildungs- und Tabellenverzeichnis ins Toc %
  ngerman,%             Weiterreichen "Neue Rechtschreibung" an weiter Pakete %
  headsepline=0.08em,%
]{scrbook}%                   DiplA aufgebaut auf der Koma-Script Buch-Klasse %

\input{preamble/layout.tex}
% \defbibheading{bibintoc}[\bibname]{%
%   \addchap{#1}}
\input{preamble/commands.tex}

\AtBeginDocument{%
  \renewcommand*{\subsectionautorefname}{\sectionautorefname}%
  \renewcommand*{\subsubsectionautorefname}{\subsectionautorefname}%
  \renewcommand*{\paragraphautorefname}{\subsubsectionautorefname}%
  \renewcommand*{\subparagraphautorefname}{\paragraphautorefname}%
  \renewcommand*{\lstlistlistingname}{Quellcodeverzeichnis}%
}
% \newcommand{\todo}[1]{\marginpar{\color{red}#1}}
\newcommand{\todotext}[1]{\todo[inline]{#1}}
\newcommand{\note}[1]{\todo{{#1}}}

\hyphenation{}
\begin{document}

\frontmatter

\input{preamble/title}


\cleardoublepage
\pdfbookmark{\contentsname}{Inhaltsverzeichnis}
\tableofcontents

% %-----------------------------------------------------------------------------%
% % Verzeichnisse. Bitte leere Verzeichnisse auskommentieren.                   %
% %-----------------------------------------------------------------------------%
\listoffigures
% \listoftables
% %\lstlistoflistings
\chapter*{Abkürzungsverzeichnis}\markboth{}{Abkürzungsverzeichnis}
\addcontentsline{toc}{chapter}{Abkürzungsverzeichnis}
\begin{acronym} % Dort, wo "JSON" steht, kommt die längste Abkürzung hin, damit wird die Ausrichtung definiert.
% Bitte die Abkürzungen alphabetisch sortieren. Das muss manuell geschehen.
% Achtung! Nur wirklich verwendete Abkürzungen erscheinen im Abkürzungsverzeichnis.
    \acro{BI}[BI]{Business Intelligence}
    \acro{API}[API]{Application Programming Interface}
\end{acronym}

%-----------------------------------------------------------------------------%
% Sperrvermerk. Bitte möglichst nicht verwenden.                              %
%-----------------------------------------------------------------------------%


%-----------------------------------------------------------------------------%
% Hauptteil                                                                   %
%-----------------------------------------------------------------------------%
\mainmatter

% Alle Abkürzungen zurücksetzen, damit Abkürzungen, die im Abstract verwendet wurden
% nochmal formal eingeführt werden.
% \acresetall
%Literatur einfügen     \cite[S.3]{bibkey} 
% In JabRef die zu zitierende Quelle auswählen und STRG+L (z.B. bei Verwendun von TeXstudio) oder STRG+K (hier wird direkt "\cite{bibkey}" kopiert) drücken. Dann mit STRG+V an der Stelle einfügen, an welcher man den Literaturverweis einfügen möchte.
%\section*{Spervermerk} Nicht sichtbares Kapitel in der Gliederung

% In TeXstudio
% Strg+T  Kommentiert Abschnitte oder Zeilen aus 
% Strg+U  Entfernt die auskommentierung


\chapter{Einleitung}
\label{sec:Einleitung} %Textmarke/Positionsmarke, um mit Autoref darauf zu verweisen.

Die Geschichte der \acl{BI} reicht weiter in die Vergangenheit zurück als moderne Informationssysteme, obwohl die beiden Begriffe in gegenwärtigen Diskussionen oft synonym miteinander verwendet werden.\todotext{Quelle: Historische Fragmente einer Integrationsdisziplin – Beitrag zur Konstruktgeschichte der Business Intelligence}
Vor dem Hintergrund, dass alle Wirtschaftssektoren jedoch mittlerweile mit IT-Systemen verwickelt sind und bestimmte Branchen wie das Gesundheits- oder Finanzwesen überwältigend von diesen abhängig sind\todotext{Quelle: https://www.sciencedirect.com/science/article/pii/S0166497222001304}, ist die Auswertung der Effizienz und Resilienz von betrieblichen Prozessen durch Computer und Netzwerke nicht mehr
wegzudenken.
Damit \acs{BI}-Lösungen den aktuellen Stand von Business Prozessen abbilden können, müssen sie mit der technischen Implementierung der Prozesse kompatibel sein. In diesem Bereich hat sich in den letzten 20 Jahren einiges getan.
Entwickler und Anwendungsadministratoren konnten in diesem Zeitraum viele Paradigmenveränderungen durch virtuelle Maschinen, Containerisierung und Clouddienste beobachten und zu diesen beitragen.
Die moderne IT-Infrastrukturlandschaft ist vor allem seit der Veröffentlichung von Container-Engines wie Docker und Container-Orchestrierungsplattformen wie Kubernetes eine komplett andere wie zu Hochzeiten der Bare-Metal-Server, Terminalserver und Mainframes.

Um mit diesen Veränderungen schritthalten zu können, haben BI-Anwendungen wie Prometheus(Metriken, 2012), Grafana(Dashboards, 2014) oder Metabase(Datenbankvisualisierung, 2015) vorgefertigte Formate für diese neueren Plattformen veröffentlicht.
Metabase bietet neben dem eigenen Clouddienst direkt Anleitungen für Self-Hosting per Docker Container. Prometheus und Grafana sind sogar in einem gemeinsamen "Helm-Chart" namens "kube-prometheus-stack" verpackt. Helm ist ein Paketmanager für Kubernetes Anwendungen und vereinfacht es Nutzern komplexe Anwendungen mit mehreren Kubernetes Ressourcen zu installieren.
Alle diese Produkte sind aber grundlegend plattformagnostisch, was zwar zu mehr Flexibilität aber auch geringerer Intergration mit der Plattform führt. Da Kubernetes als modernes Produktivsystem jedoch eine professionelle und frei erweiterbare \acs{API} bietet, kann diese potenziell genutzt werden um direkt BI-Lösungen zu deklarieren.

Ein weiterer Schwachpunkt von etablierten BI-Lösungen, der vor Allem bei Prometheus und Grafana auffällt, ist die hohe Komplexität bei der Erstellung von übersichtlichen Dashboards.
Prometheus sammelt Metriken aus sämmtlichen Anwendungen die diese standardmäßig anbieten, jedoch wird hierbei keine Beschreibung mitgeliefert, was die Metriken genau bedeuten oder wie diese bestenfalls in einem Dashboard visualisiert werden können.
Falls also kein fertiges Dashboard von den Entwicklern einer Anwendung mitgeliefert wird, fällt die Arbeit dieses zu entwickeln auf die Endnutzer.
Da Teammitglieder, die diese Metriken für geschäftsbezogene Entscheidungen verwenden nicht immer speziell geschult sind, kommen Infrastrukturadminstratoren oder ein DevOps-Team für technische Defizite auf und implementieren die Visualisierungen.
Idealerweise ist die Überwachung von Anwendungen so einfach präsentiert, dass ungeschultes Personal die Erstellung von Dashboards intuitiv lernen kann.\todotext{Quelle: Microsoft Power BI oder so}

\section{Problemstellung}
Da Kubernetes ein beliebtes Kompilierungsziel für Business Intelligence Anwendungen ist, aber diese Anwendungen die Möglichkeiten einer tiefen Integration, die durch den offenen Quellcode erreichbar ist, nicht ausnutzen, soll in dieser Arbeit durch eine Fallstudie untersucht werden ob eine direkte Erweiterung der Kubernetes API dabei helfen kann eine intuitive Datenwertschöpfung für Endnutzer zu ermöglichen.

Folgende Forschungsfrage soll in dieser Arbeit beantwortet werden:
"`Können im Cloudnative Bereich Self-Service Paradigmen der Anwendungsentwicklung auf Business Intelligence ausgeweitet werden um Endnutzern eine einfachere Erfahrung bei der Überwachung von Diensten zu ermöglichen`"


\section{Lösungsansatz}
\label{sec:loesungsansatz}
Durch explorative Auseinandersetzung mit Design Mustern für Kubernetes-Erweiterungen, soll überprüft werden ob Best Practices bei deren Entwicklung dabei helfen können eine flüssigere und simplere BI-Pipeline zu entwickeln.
Es soll festgestellt werden ob Nutzern durch weniger Abstraktionsschichten die Erstellung von Dashboards vereinfacht werden kann.
Die Ergebnisse dieser Fallstudie sollen dabei helfen zukünftige Monitoringlösungen für Kubernetes und ähnliche Orchestrierungsplattformen zu konzipieren.

\section{Aufbau der Arbeit}
Vorerst werden grundlegende Begriffe aus dem Containerökosystem erklärt und darauf aufgebaut, wie aktuelle BI-Lösungen mit ihren Aufgaben umgehen.
Anschließend wird anhand einer Fallstudie eine beispielhafte Anwendung geschrieben, die nachfolgend auf Schwachstellen und Vorteile bewertet wird.
Zuletzt werden die daraus resultierenden Ergebnisse kritisch gewürdigt und die Forschungsfrage bestätigt oder widerlegt.


\chapter{Theoretischer Hintergrund}
\label{sec:Hintergrund}

Obwohl die Systeme rund um "`Container"' in vielen Unternehmen weltweit eingesetzt werden und vor Allem bei IT-Firmen als der Standard für Anwendungsadministration gesehen werden \todotext{Quelle: https://www.docker.com/blog/2025-docker-state-of-app-dev/}, kann deren Komplexität und Umfang für Außenstehende überfordernd sein.
Um die Fallstudie dieser Arbeit in den richtigen Kontext zu bringen, werden nachfolgend einige Fachbegriffe aus der modernen Softwareentwicklung und Anwendungsadministration beschrieben.

\section{Cloudnative Anwendungslandschaft}

Die "`\acl{CNCF}"'\todotext{Quelle: https://www.cncf.io/about/who-we-are/} definiert Cloudnative Anwendungen wie gefolgt:
\begin{displayquote}
Cloud native Technologien ermöglichen es Unternehmen, skalierbare Anwendungen in modernen, dynamischen Umgebungen zu implementieren und zu betreiben. Dies können öffentliche, private und Hybrid-Clouds sein. Best Practices, wie Container, Service-Meshs, Microservices, immutable Infrastruktur und deklarative APIs, unterstützen diesen Ansatz.
\end{displayquote}\todotext{Quelle: https://github.com/cncf/toc/blob/main/DEFINITION.md#deutsch}

Diese Technologien sind vielfältig und werden durch die \acs{CNCF} in unterschiedlichen Phasen basierend auf Reifegraden finanziell und organisatorisch unterstützt.
Der Fokus bei diesen Anwendung liegt auf der Betreibung von Microservices.

\subsection{Microservices}
Als Gegenstück zu "Monolith"-Anwendungen, bei denen alle Teile einer Anwendung zu einem Paket zusammengesteckt und gestartet werden, bieten Microservices die Möglichkeit eine Anwendung in kleinere Bestandteile zu zerlegen.
Vorteile hierbei sind eine effektivere Skalierbarkeit, Fehlerisolation und Flexibilität in Teams.
Dadurch dass diese Anwendungsform standardmäßig jedem Programmteil seine eigene Umgebung und eigene Rechenressourcen reserviert, können diese Ressourcen individuell vertikal und horizontal skaliert werden und behindern nicht die Prozessorleistung oder den Arbeitsspeicherplatz der anderen Teile.
Eine Verteilung einer Anwendung auf mehrere Hardwareserver, was im Falle von Kubernetes einfach umsetzbar ist, bietet außerdem niederschwellige Redundanzen für Ausfallsicherheit.
Unterschiedliche Microservices können außerdem in unterschiedlichen Programmiersprachen geschrieben werden, was Entwicklerteams erlaubt aus Erfahrung zu unterschiedlichen Domänen, Sprachen und Bibliotheken zu schöpfen.

\subsection{Containerumgebungen}
Die Laufzeitumgebungen von Microservices basieren zum größten Teil auf Linux-Containern. WebAssembly und das WebAssembly System Interface(WASI) bieten eine Alternative, sind jedoch deutlich weniger verbreitet.


\subsection{Kubernetes}

\section{Open Source BI-Lösungen}


\chapter{Fallstudie}

\section{Datenmodell}

\section{Kubernetes Operator}

\section{Benutzerinterface}


\chapter{Einordnung und Bewertung der Erkenntnisse}


\chapter{Kritische Würdigung}


%-----------------------------------------------------------------------------%
% Anhang                                                                      %
%-----------------------------------------------------------------------------%
\appendix

\chapter{Google Platform Status}
\label{gps}

BigQuery-Abfrage:
\begin{verbatim}    
#standardSQL
SELECT date, client, pct_urls, sample_urls
FROM `httparchive.blink_features.usage`
WHERE feature = 'WebAssemblyInstantiation'
AND date = (
  SELECT MAX(date) 
  FROM `httparchive.blink_features.usage`
)
ORDER BY date DESC, client
\end{verbatim}

\chapter{Konferenzvorträge}
\label{conf}

\begin{tabular}{ m{0.8cm} m{2.5cm} m{3cm} m{6cm}}
    Jahr & Konferenz & Vortragende Person & Titel \\
    \hline
    2017 & dotJS &  Sean Larkin & {Webpack+WebAssembly:\newline Under the hood} \\
    & JSConf EU &  Lin Clark & A Cartoon Intro to WebAssembly \\
    & JSConf EU &  Dan Callahan & Practical WebAssembly \\
    & GOTO &  Ben Smith & We Want WebAssembly \\
    \hline
    2018 & dotJS &  Tejas Kumar &  A Quick Recap on WebAssembly \\
    & JSConf EU &  Emil Bay & Hand-crafting WebAssembly \\
    & JSConf EU &  Lin Clark & {Baby’s First Rust+WebAssembly \newline module} \\
    & JSConf Asia &  David Bryant & Enabling New Web Experiences \\
    & JSUnconf & Johann Hofmann & Putting WebAssembly in your web app today! \\
    & GOTO &  Lin Clark & A Cartoon Quest: New Adventures for WebAssembly \\
    \hline
    2019 & dotJS &  Vlad Filippov & Into WebAssembly \\
    & dotJS & Sven Sauleau & More WebAssembly in your JavaScript \\
    & JSConf EU & Max Bittker & Simulating Sand: Building Interactivity With WebAssembly \\
    & JSConf Korea & Istvan Szmozsanszky & A WebAssembly Field Guide easily worth like 70 bottle caps \\
    & JSConf Asia & Kas Perch & WebAssembly: The Future of JS and a Multi-Language Web \\
    & JSConf US & Florian Rival & {Native Web Apps: React, JS \& \newline WebAssembly to rewrite native apps} \\
    & JSConf Hawaii & {Lin Clark /\newline Till Schneidereit} & New Adventures for WASM \\
    & GOTO & Dan Callahan & WebAssembly Beyond the Browser \\
    \hline
    2023 & GOTO & Brian Carroll & {WebAssembly in Production:\newline A Compiler in a Web Page} \\
    & YOW! & Katie Bell & {Don't Trust Anything!\newline Real-world Uses For WebAssembly} \\
    \hline
    2024 & dotJS & David Flanagan & {The Future of Serverless is \newline WebAssembly}
\end{tabular}

\chapter{R-Skript}
\label{sec:rSkript}

% \lstinputlisting[language=R]{skript.R}


\backmatter
%-----------------------------------------------------------------------------%
% % Literaturverzeichnis                                                        %
% %-----------------------------------------------------------------------------%
\cleardoublepage
% \printbibliography
% Hier kann das Literaturverzeichnis noch getrennt werden.
% Dafür ist zwingend biblatex notwendig!
\ohead[]{Literatur}
\printbibliography[nottype=misc]

\printbibliography[type=misc,notkeyword=Experteninterview,title={Sonstige Quellen}]

\cleardoublepage
\ohead[]{\headmark}

\end{document}
